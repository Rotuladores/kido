\documentclass[12pt,italian]{article}
%----------------------------
%		PACKAGES
%----------------------------
\usepackage{fullpage}
\usepackage[T1]{fontenc}
\usepackage[utf8]{inputenc}
\usepackage{babel}
\usepackage{latexsym}
\usepackage[leqno]{amsmath}
\usepackage{amssymb}
\usepackage{amstext}
\usepackage{xcolor}
\usepackage{graphicx}
\usepackage[displaymath, mathlines]{lineno}
\usepackage{authblk}
\usepackage{multicol}
\usepackage{tikz}
\usetikzlibrary{trees,positioning,shapes} 
% for listing source code files
\usepackage{listings}
\usepackage{hyperref}
\usepackage{float}
\restylefloat{table}

%----------------------------
%			TITLE
%----------------------------
\title{Kido, an HMM-driven spelling corrector}
\author[1]{Simone Ciccolella}
\affil[1]{762234, s.ciccolella@campus.unimib.it}
\author[2]{Federica Adobbati}
\affil[1]{764300, f.adobbati@campus.unimib.it}
\author[3]{Daniele Bellani}
\affil[3]{780675, d.bellani1@campus.unimib.it}
\author[4]{Francesco Canonaco}
\affil[4]{781239, canonaco@harvard.edu, Full Professor at Harvard e capogruppo di ricerca presso Rotuladores}
\date{\today}
\setcounter{Maxaffil}{0}
\renewcommand\Affilfont{\itshape\small}
%----------------------------
%	DEFINITIONS
%----------------------------
\newtheorem{defnn}{Definition}

%----------------------------
%	DOCUMENT SETTINGS
%----------------------------
\setlength\linenumbersep{20pt}
\renewcommand\linenumberfont{\normalfont\footnotesize\color{gray}}

%% Rimuovere per la sottomissione
 \linenumbers


\usepackage{textcomp}\usepackage{xcolor}
\newcommand{\notaestesa}[2]{%
  \marginpar{\color{red!75!black}\textbf{\texttimes}}%
  {\color{red!75!black}%
    [\,\textbullet\,\textsf{\textbf{#1:}} %
    \textsf{\footnotesize#2}\,\textbullet\,]}%
}
\newcommand{\SC}[1]{\notaestesa{SC}{#1}}
\newcommand{\DB}[1]{\notaestesa{DB}{#1}}
\newcommand{\FA}[1]{\notaestesa{FA}{#1}}
\newcommand{\FC}[1]{\notaestesa{FC}{#1}}
%% Fine rimozione

\begin{document}
\maketitle
%\begin{multicols}{2}                                                
\textbf{Kido} \`e un correttore di testo in lingua inglese
basato su \textit{Hidden Markov Models} e in particolare 
sull'algoritmo di Viterbi. Il suo principio \`e ispirato al 
correttore in \cite{7203984} in cui viene presentato un correttore
che utilizza Viterbi e l'algoritmo di \textit{Forward-Backward} 
in una HMM generata ad hoc per la correzzione. 
Prendendo spunto dal suddetto articolo e da diversi algoritmi
di confronto di stringhe abbiamo implementato un modello che,
sebbene necessiti di ulteriore lavoro e completamento, ha gi\`a
dei risultati molto promettenti, comparabili a lavori presenti
in letteratura e, in alcuni casi, anche migliori.
Inoltre, grazie al modo in cui \`e implementato, Kido \`e anche in 
grado, ad ogni parola inserita, di suggerire all'utente tre possibili
correzioni e aggiustare le probabilit\`a in coseguenza all'intervento
dell'utente.


Abbiamo implementato \textbf{Kido} con Python 3 nella sua interezza,
per quando riguarda il core del programma
 (\textit{SmartDictionary} e \textit{HMM}) \`e sufficiente che i file
ad esso relativo siano presenti nella stessa directory del programma.
Per poter utilizzare il programma da terminale sono necessarie le
seguenti dipendenze:

\begin{itemize}
  \item Python 3.5.2 +, \url{https://www.python.org}
  \item edlib 1.1.2.post2 +, \url{https://pypi.python.org/pypi/edlib}
  \item NumPy 1.12.1 + \url{http://www.numpy.org}
  \item BioPython 1.69 +, \url{https://github.com/biopython/biopython.github.io/}
\end{itemize}

Per l'interfaccia grafica abbiamo deciso di utilizzare il framework 
multiplatform \textbf{Kivy} per Python, reperibile all'indirizzo
\url{https://kivy.org/} che permette una forte customizzazione 
dell'intera interfaccia oltre alla portabilit\`a su tutti i sistemi
operativi, mobile inclusi. Di seguito mostriamo l'immagine
di una classica esecuzione di Kido:\\

\begin{center}
  \includegraphics[scale=.65]{img/interfaccia}
\end{center}


\section{Metodi}

\subsection{Smart Dictionary}

\subsection{HMM Custom Class}

\subsection{Custom Viterbi}


\section{Training}

\subsection{Training Set}

\subsection{Considerazioni}

\section{Testing}

\subsection{Testing Set}

\subsection{Risultati}

\section{Sviluppi futuri}

\bibliography{biblio}{}
\bibliographystyle{plain}

%\end{multicols}
\end{document}
